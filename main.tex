%
% ==============================================================================================
%                                      Thesis Template
% ==============================================================================================
% 
% - This template is for the use of the Final Year Undergraduates of the Department of          %   Statistics & Computer Science at University of Kelaniya for preparing their thesis for the 
%   Research Project.
%
% - Please note that this is by no means official and merely created by the student(s) 
%   themselves. 
% 
% - The department bears no responsibility or liability regarding this template and may change 
%   depending on the requirements of the department.
% - You may use this to craft your research thesis under your own discretion.
% 
% - Authored by Chandeepa Dissanayake (https://github.com/chandeepadissanayake/)
% 
%
% ==============================================================================================
% Instructions for Usage
%
% 1. Update the contents of the following pages as per the instructions found in
%    them
%       1.1 Thesis Information                                : thesis_info.tex
%       1.2 Personnel Information (involved with the research): personnel_info.tex
%
% 2. Add contents to the following pages with your own, customized writings.
%       2.1 Dedication:  pages/dedication.tex
%       2.2 Acknowledgement: pages/acknowledgements.txt
%
% 3. Add definitions of terms that you anticipate on using in the thesis in definitions.tex
%       - The instructions have been provided in the file
%
% 4. Add the list of abbreviations that you anticipate on using in the thesis in 
%    abbreviations.tex
%       - The instructions have been provided in the file
%
% 5. Write the abstract of your work in research/abstract.tex
%       - Instructions and guidelines are found in the file
%
% 6. Craft the contents in the following files which correspond to your research work.
%       6.1 Introduction : research/introduction/index.tex
%           6.1.1 Background   : research/introduction/background.tex
%           6.1.2 Objectives   : research/introduction/objectives.tex
%           6.1.3 Significance : research/introduction/significance.tex
%           
%           - Any other sub-sections could be added as guided in the file.
%       6.2 Literature Review: research/literature_review/index.tex
%           - Any sub-sections could be added as guided in the file.
%       6.3 Methodology: research/methodology/index.tex
%           - Any sub-sections could be added as guided in the file.
%       6.4 Results and Discussion: research/results_and_discussion/index.tex
%           - Any sub-sections could be added as guided in the file.
%       6.5 Conclusions: research/conclusions/index.tex
%           6.5.1 Limitations: research/conclusions/limitations.tex
%           6.5.2 Future Work: research/conclusions/future_work.tex
%           
%           - Any other sub-sections could be added as guided in the file.
%       6.6 Appendices: research/appendices/index.tex
%           - Follow the in-file instructions for adding appendices
%
%       - Following guidelines must be followed when compiling the contents in the pages under
%         6
%            I) Tables:
%                 i. Use \table tag to define all tables. Borders must be visible and 1pt.
%                ii. Headers must be in bold face.
%               iii. Tables must be captioned and centered.
%                iv. Padding must be such that the borders don’t touch the text in every cell.
%                 v. Text in each cell must be left-aligned (flushed left)
%           II) Figures:
%                 i. Use \figure tag to define all figures.
%                ii. Figures must be captioned.
%               iii. All figures must ONLY be uploaded to resources/figures folder.
%          III) References:
%                 i. All reference material that are used must be defined in 
%                    research/references.tex. This can be achieved in either of the following 
%                    methods.
%                ii. Defined as a list in references.tex using \bibitem. Refer to: 
%                    https://www.overleaf.com/learn/latex/Bibliography_management_with_bibtex
%               iii. Defined in a .bib file, typically exported from a reference management 
%                    tool. Steps are as follows.
%                       a. Upload all .bib files into resources/references folder.
%                       b. Add all files as references into research/references.tex file using
%                          the tag: \bibliography{bib_file_name.bib}
%
% Important NOTE:  
%   Add any packages that you need to use in your compilation in settings/user_dependencies.tex
%
% Additional Notes:
%
%   I. Do NOT attempt to change the contents of portions of markup where it is explicitly
%      specified not to do so, unless you're assured about what you're doing.
%  II. Contents inside settings/global.tex is better left intact. Changing them would cause 
%      large and possibly catastrophic changes to the styles.
% III. Do not change any template dependencies in settings/template_dependencies.tex as well.
%  IV. It's better not to change the folder structure.
%   V. You can add any custom environments inside the folder envs but be cautious in doing so.
%
% ==============================================================================================
\documentclass[12pt, openany]{book} % openany for eliminating the blank pages in between chapters

% ==========================================================
% Template Dependencies
% DO NOT change.
% Following are the template dependencies.

\usepackage{fancyhdr} % For header customization
\usepackage{titlesec} % For formatting chapter titles
\usepackage[a4paper, total={6in, 8in}, margin=1.0in]{geometry} % Page Setup
\usepackage{graphicx} % Required for inserting images
\usepackage{times} % Times New Roman font
\usepackage{parskip} % For formatting paragraphs
\usepackage{setspace} % For line spacing
\usepackage{ragged2e} % For text alignment
\usepackage{tabularx} % For table-like and table structures in the document
\usepackage{tocloft} % For formatting (adding dots) in table of contents
\usepackage[acronym, toc]{glossaries} % For definitions and abbreviations
\usepackage[style=ieee]{biblatex} % For references
\usepackage[toc]{appendix} % For appendices


% Template Dependencies
% Following are the user dependencies.

% Add any dependencies that you need on the go here.
% Use the tag \usepackage to add any dependencies


% Global Settings
% DO NOT change unless specified explicitly

% Cleaning up default document class's headers and footers.
\fancypagestyle{plain}{
    \lhead{}
    \fancyhead[R]{}
    \fancyhead[L]{}
    \renewcommand{\headrulewidth}{0pt}
    \fancyfoot[C]{\thepage}
}
\pagestyle{fancy}
\fancyhead[R]{}
\fancyhead[L]{}
\renewcommand{\headrulewidth}{0pt}
\fancyfoot[C]{\thepage}

% Chapter Title Formatting
\titleformat{\chapter}[hang]
{\normalfont\fontsize{16}{19.2}\bfseries}
{\thechapter.}
{1em}
{}

\titleformat{\section}
{\normalfont\fontsize{16}{19.2}\bfseries}
{\thesection}
{1em}
{}

\titlespacing*{\chapter}{0pt}{0pt}{0.5\baselineskip}

% Other settings
\parindent 0pt
\doublespacing
\graphicspath{ {resources/figures/} }
\renewcommand{\cftsecleader}{\cftdotfill{\cftdotsep}} % Add dots to table of contents, other other similar pages
\makeglossaries

% Including environments
% DO NOT change

% DO NOT change.

\newcommand{\nonresearchchapter}[1]{
    \chapter*{#1}
    \addcontentsline{toc}{chapter}{#1}
}

% If you really need to add content here, use a separate file and include it below using \input
% tag.

% Definitions
% NOTE: All the definitions of the terms used throughout the thesis must strictly be defined
%       here.

% Write the definitions in the following format in this page
%
% \newglossaryentry{latex}
% {
%         name=latex,
%         description={Is a mark up language specially suited for scientific documents}
% }
% 
% Such definitions can then be used in any page as:
%
% \Gls{latex}
%


% Abbreviations
% NOTE: All the abbreviations used throughout the thesis must strictly be defined here.

% Write the abbreviations in the following format in this page
%
% \newacronym{gcd}{GCD}{Greatest Common Divisor}
%
% Such abbreviations can then be used in any page as:
%
% \acrlong{gcd}
%
% or,
%
% \acrshort{gcd}
%
% or,
%
% \acrfull{gcd}
%


% References
\input{research/references}
% ==========================================================

\begin{document}

    % DO NOT change the content below unless you are totally aware about what you're doing.

% Change the contents appropriately.

% Title of your work/thesis/research
\newcommand{\thesisTitle}{THESIS TITLE GOES HERE}
% Your full name, DO NOT use initials
\newcommand{\thesisAuthor}{THESIS AUTHOR'S NAME GOES HERE}
% The degree programme you're currently following
% For Science Students: "BACHELOR OF SCIENCE HONOURS"
% For Arts Students:    "BACHELOR OF ARTS HONOURS"
\newcommand{\degreeProgramme}{DEGREE PROGRAMME TITLE GOES HERE}
% The specialization of your degree programme
% For COSC:    "Computer Science"
% For COST:    "Computer Studies"
\newcommand{\degreeSpecialization}{SPECIALIZATION GOES HERE}
% The year in which you are publishing the thesis
\newcommand{\thesisYear}{YEAR GOES HERE}


\begin{titlepage}
  \begin{center}
    \vspace*{0.5cm}

    \textbf{\thesisTitle}

    \vspace{0.9cm}

    By\\[1\baselineskip]
    \thesisAuthor{}

    \vfill
    \begin{onehalfspace}
        A THESIS\\
        Submitted in partial fulfillment of the requirements for the degree of\\
        \degreeProgramme{}\\
        In \degreeSpecialization
    \end{onehalfspace}

    \vspace{9.4cm}

    DEPARTMENT OF STATISTICS \& COMPUTER SCIENCE\\
    UNIVERSITY OF KELANIYA, SRI LANKA.\\
    \thesisYear{}

    \vspace*{0.5cm}

  \end{center}
\end{titlepage} \clearpage
    % Change the contents appropriately.

% ==============================================================================================
% Candidate Information
% =======================

% Your full name. DO NOT use initials
\newcommand{\candidateFullName}{CANDIDATE FULL NAME GOES HERE}
% Your name with initials.
\newcommand{\candidateInitialsName}{CANDIDATE NAME WITH INITIALS GOES HERE}

% ==============================================================================================

% ==============================================================================================
% SUPERVISOR Information
% =======================

% Supervisor's Name with Initials, for example: Dr. S. P. Pitigala
\newcommand{\supervisorInitialsName}{SUPERVISOR NAME WITH INITIALS GOES HERE}
% Supervisor's Designation, for example: Senior Lecturer
\newcommand{\supervisorDesignation}{SUPERVISOR DESIGNATION GOES HERE}
% Supervisor's Affiliated Department, for example: Department of Statistics & Computer Science
\newcommand{\supervisorAffiliationDepartment}{SUPERVISOR'S AFFILIATED DEPARTMENT GOES HERE}
% Supervisor's Affiliated University, for example: University of Kelaniya
\newcommand{\supervisorAffiliationUniversity}{SUPERVISOR'S AFFILIATED UNIVERSITY GOES HERE}

% ==============================================================================================

% ==============================================================================================
% Coordinator Information
% =======================

% Coordinator's Name with Initials, for example: Dr. S. P. Pitigala
\newcommand{\coordinatorInitialsName}{COORDINATOR NAME WITH INITIALS GOES HERE}
% Coordinator's Designation, for example: Senior Lecturer
\newcommand{\coordinatorDesignation}{COORDINATOR DESIGNATION GOES HERE}
% Coordinator's Affiliated Department, for example: Department of Statistics & Computer Science
\newcommand{\coordinatorAffiliationDepartment}{COORDINATOR'S AFFILIATED DEPARTMENT GOES HERE}
% Coordinator's Affiliated University, for example: University of Kelaniya
\newcommand{\coordinatorAffiliationUniversity}{COORDINATOR'S AFFILIATED UNIVERSITY GOES HERE}

% ==============================================================================================

\begin{center}
    \textbf{DECLARATION}
\end{center}

\vspace{0.6cm}

This thesis is my original work and has not been submitted previously for a degree at this or any other university/institute. To the best of my knowledge, it does not contain any material published or written by another person, except as acknowledged in the text.\par

\vspace{0.8cm}

\begin{tabularx}{\linewidth}{lXr}
    Candidate: & \candidateInitialsName & Signature: …………………..\\
    & & Date: …………………..\\

    & & \\

    Supervisor: & \supervisorInitialsName & Signature: …………………..\\
    & \supervisorDesignation, & \\
    & \supervisorAffiliationDepartment, & \\
    & \supervisorAffiliationUniversity, & \\
    & & Date: …………………..\\

    & & \\

    Coordinator: & \coordinatorInitialsName & Signature: …………………..\\
    & \coordinatorDesignation, & \\
    & \coordinatorAffiliationDepartment, & \\
    & \coordinatorAffiliationUniversity, & \\
    & & Date: …………………..\\
\end{tabularx} \clearpage
    % DO NOT change the content below unless you are totally aware about what you're doing.

% Changing Contents Table page
\renewcommand*\contentsname{Table of Contents}
\singlespacing
\tableofcontents
\doublespacing \clearpage
    
    % Disable counting for the following sections
    \setcounter{secnumdepth}{0}
    % DO NOT change the content below unless you are totally aware about what you're doing.

\renewcommand{\listfigurename}{}

\nonresearchchapter{List of Figures}
    \thispagestyle{empty}
    \listoffigures \clearpage
    % DO NOT change the content below unless you are totally aware about what you're doing.

\renewcommand{\listtablename}{}

\nonresearchchapter{List of Tables}
    \thispagestyle{empty}
    \listoftables
 \clearpage
    \nonresearchchapter{Dedication}

% Write the dedication here
% A possible paragraph may be started off as:
%   "I would like to dedicate this thesis to everyone who supported me to complete the study 
%    successfully" \clearpage
    \nonresearchchapter{Acknowledgements}

% Write the acknowledgements here
% This page is used to thank individuals, groups, or organizations for their support.  If you 
% are required to acknowledge the support of a sponsor, this is generally an appropriate place 
% to do so.
 \clearpage
    % NOTE: Remove the clearpage tag in the following pages based on whether there is no 
    % definition/no abbreviation in your thesis.
    % DO NOT change the content below unless you are totally aware about what you're doing.
% This page will only appear if there are any definitions and they are being used in any of the 
% pages

\printglossary[title=Definitions]
 \clearpage
    % DO NOT change the content below unless you are totally aware about what you're doing.
% This page will only appear if there are any definitions and they are being used in any of the 
% pages

\printglossary[type=\acronymtype, title=List of Abbreviations] \clearpage

    % DO NOT change the following lines
\nonresearchchapter{Abstract}

% Write down your abstract here. Here's something that could help:
%   An abstract is a concise summary of the document. It is not an introduction. It should 
%   clearly identify the topic and major findings of the research. It should be written in a 
%   single page, and it should not be extended for multiple pages. 
 \clearpage
    
    % Enable counting for the remaining sections
    \setcounter{secnumdepth}{1}

    % Research Pages
    \chapter{Introduction}

% Write the content of the introduction here. Here's something that could help.
%   This section provides an overall introduction to your study. 

% DO NOT modify the following lines

\section{Background}
    % Write the content of the background of the research here. Here's something that could help.
%   This subsection outlines the historical development in the literature led to the current 
%   topic of research.
\clearpage

\section{Objectives}
    % Write the content of the objectives of the research here. Here's something that could help.
%   The primary and the subsidiary objectives need to be present in this section.
\clearpage

\section{Significance of the Study}
    % Write the content of the significance of the research here. Here's something that could help.
%   This section presents the significance of the study.
\clearpage

% Add any other required sub-sections here.
% Do not fill the contents of the sub-sections here itself. Use a separate file created in
%   research/introduction.
% Use the following format to add any such sub-sections:
%   \section{SUBSECTION TITLE GOES HERE}
%       \input{SUBSECTION FILE PATH GOES HERE}
%   \clearpage

    % DO NOT change
\chapter{Literature Review}

% Write the content of the literature review here. Here's something that could help.
%   Present the relevant literature for your study. Discuss the strengths and weaknesses of the 
%   past studies.

% Add any other required sub-sections here.
% Do not fill the contents of the sub-sections here itself. Use a separate file created in
%   research/literature_review.
% Use the following format to add any such sub-sections:
%   \section{SUBSECTION TITLE GOES HERE}
%       \input{SUBSECTION FILE PATH GOES HERE}
%   \clearpage

    % DO NOT change
\chapter{Methodology}

% Write the content of the methodology here. Here's a guideline for a flow that you could use.
%   1. Data description
%   2. Conceptual Map
%   3. Implementation Details
%   4. Validation and etc.

% Add any other required sub-sections here.
% Tip: If it seems useful, use the aforementioned flow as sub-sections.
% Do not fill the contents of the sub-sections here itself. Use a separate file created in
%   research/methodology
% Use the following format to add any such sub-sections:
%   \section{SUBSECTION TITLE GOES HERE}
%       \input{SUBSECTION FILE PATH GOES HERE}
%   \clearpage

    % DO NOT change.
\chapter{Results and Discussion}

% Write the content of the results and discussion here. Here's something that could help.
%   Summary of your results with relevant discussion.

% Add any other required sub-sections here.
% Do not fill the contents of the sub-sections here itself. Use a separate file created in
%   research/results_and_discussion
% Use the following format to add any such sub-sections:
%   \section{SUBSECTION TITLE GOES HERE}
%       \input{SUBSECTION FILE PATH GOES HERE}
%   \clearpage
    % DO NOT change
\chapter{Conclusions}

% Write the content of the conclusions here. Here's something that could help.
%   Clearly presents the conclusions of the study.

% DO NOT modify the following lines

\section{Limitations}
    % Write the content of the limitations here. Here's something that could help.
%   Present the limitations of your study.

% Add any other required sub-sections here.
% Do not fill the contents of the sub-sections here itself. Use a separate file created in
%   research/conclusions
% Use the following format to add any such sub-sections:
%   \section{SUBSECTION TITLE GOES HERE}
%       \input{SUBSECTION FILE PATH GOES HERE}
%   \clearpage

\clearpage

\section{Future Work}
    % Write the content of the future work here. Here's something that could help.
%   Propose future research that would improve present work in this area.

% Add any other required sub-sections here.
% Do not fill the contents of the sub-sections here itself. Use a separate file created in
%   research/conclusions
% Use the following format to add any such sub-sections:
%   \section{SUBSECTION TITLE GOES HERE}
%       \input{SUBSECTION FILE PATH GOES HERE}
%   \clearpage

\clearpage

% Add any other required sub-sections here.
% Do not fill the contents of the sub-sections here itself. Use a separate file created in
%   research/conclusions
% Use the following format to add any such sub-sections:
%   \section{SUBSECTION TITLE GOES HERE}
%       \input{SUBSECTION FILE PATH GOES HERE}
%   \clearpage

    % DO NOT change the content below unless you are totally aware about what you're doing.

\chapter{Reference List}
    \printbibliography[heading=none]

    % All the appendices go here.
% 
% ==========================================================
% Instructions to add a new appendix
% 
% 1. Add a new latex page with a name to identify it uniquely 
%    in the folder: research/appendices
% 2. Add a new chapter as guided below
%
% NOTE: Do not fill the contents of the sub-sections here itself. 
%       Use a separate file created in research/appendices.
% ==========================================================
% 

\begin{appendices}
    % Add the appendices below as follows;
    %
    % \chapter{Appendix Title}
    %   \input{research/appendices/appendix_file_name}
    
    \chapter{Appendix: Test Appendix}
        hello appendix
    
\end{appendices}




\end{document}
